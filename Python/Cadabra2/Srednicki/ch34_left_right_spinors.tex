\documentclass[12pt]{article}
\usepackage{amsmath, amssymb, amsthm}
\usepackage{mathtools}
\usepackage{tensor}
\usepackage{geometry}
\usepackage{booktabs}
\usepackage{array}
\usepackage{xcolor}
\usepackage{hyperref}
\usepackage{fancyhdr}
\geometry{margin=1.1in, headheight=15pt}

\definecolor{cadblue}{RGB}{30,90,200}
\definecolor{codegray}{RGB}{245,245,245}
\definecolor{verifygreen}{RGB}{20,140,60}

\newcommand{\cdbexpr}[1]{\textcolor{cadblue}{$#1$}}
\newcommand{\verified}[1]{\textcolor{verifygreen}{\checkmark\; #1}}
\newcommand{\SL}{S^{\mu\nu}_{\mathrm{L}}}
\newcommand{\SR}{S^{\mu\nu}_{\mathrm{R}}}
\newcommand{\psidag}{\psi^{\dagger}}
\newcommand{\half}{\tfrac{1}{2}}
\newcommand{\mmat}[4]{\begin{pmatrix}#1&#2\\#3&#4\end{pmatrix}}

\pagestyle{fancy}
\fancyhf{}
\lhead{Srednicki QFT --- Chapter 34}
\rhead{Left- \& Right-Handed Spinor Fields}
\cfoot{\thepage}

\title{\textbf{Srednicki QFT: Chapter 34}\\[6pt]
       \large Left- and Right-Handed Spinor Fields\\[4pt]
       \normalsize Cadabra2 expressions and numerical verification}
\author{Generated by \texttt{ch34\_export\_latex.py}}
\date{}

\begin{document}
\maketitle
\tableofcontents
\newpage

%% ============================================================
\section{Lorentz Group Representations}
%% ============================================================

The Lorentz algebra in four dimensions is isomorphic to
$\mathfrak{su}(2)_L \oplus \mathfrak{su}(2)_R$ via the
non-Hermitian combinations
\begin{align}
  N_i &\equiv \tfrac{1}{2}(J_i - iK_i), \qquad
  N_i^\dagger \equiv \tfrac{1}{2}(J_i + iK_i),
\end{align}
satisfying $[N_i, N_j] = i\varepsilon_{ijk}N_k$,
$[N_i^\dagger, N_j^\dagger] = i\varepsilon_{ijk}N_k^\dagger$,
$[N_i, N_j^\dagger] = 0$.

Irreducible representations are therefore labelled by two numbers
$n, n' \in \{0, \tfrac{1}{2}, 1, \ldots\}$:

\medskip
\begin{center}
\begin{tabular}{@{}ccccc@{}}
  \toprule
  Srednicki label & Physics label & Dimensions & Field & Index type \\
  \midrule
  $(1,1)$ & $(0,0)$ & 1 & scalar $\phi(x)$ & none \\
  $(2,1)$ & $(\tfrac{1}{2},0)$ & 2 & left-handed Weyl $\psi_a$ & undotted \\
  $(1,2)$ & $(0,\tfrac{1}{2})$ & 2 & right-handed Weyl $\psidag_{\dot{a}}$ & dotted \\
  $(2,2)$ & $(\tfrac{1}{2},\tfrac{1}{2})$ & 4 & vector $A^\mu$ & spacetime \\
  \bottomrule
\end{tabular}
\end{center}
\medskip

\noindent\textbf{Convention note.}  Srednicki labels representations by their
\emph{dimensions} $(2n{+}1, 2n'{+}1)$.  The physics literature often uses
the spins directly, writing $(\tfrac{1}{2},0)$ for what Srednicki calls $(2,1)$.
Both label the same object.

%% ============================================================
\section{Left-Handed Spinor Field \texorpdfstring{$\psi_a$}{psi\_a}}
%% ============================================================

A left-handed Weyl field $\psi_a(x)$ lives in the $(2,1)$ representation.
Under a finite Lorentz transformation $\Lambda$:
\begin{equation}
  U(\Lambda)^{-1}\,\psi_a(x)\,U(\Lambda)
  = {L_a}^b(\Lambda)\,\psi_b(\Lambda^{-1}x).
  \tag{34.1}
\end{equation}
For an infinitesimal transformation
${\Lambda^\mu}{}_\nu = \delta^\mu{}_\nu + \delta\omega^\mu{}_\nu$:
\begin{equation}
  {L_a}^b(1{+}\delta\omega)
  = \delta_a{}^b + \tfrac{i}{2}\,\delta\omega_{\mu\nu}\,
    {(S^{\mu\nu}_{\mathrm{L}})_a}^b.
  \tag{34.3}
\end{equation}

\noindent In Cadabra2 notation, the left-handed field and its generator are:
\[
  \psi_{\alpha},
  \qquad
  \left\left(S^{\mu \nu}\,_{L} \right\right)\,_{\alpha}\,^{\beta}
\]

%% ============================================================
\section{Generators \texorpdfstring{$S^{\mu\nu}_{\mathrm{L}}$}{S\^{}munu\_L}
         in the $(2,1)$ Representation}
%% ============================================================

\subsection{Commutation relations}

The six $2\times 2$ generator matrices $S^{\mu\nu}_{\mathrm{L}}$
(antisymmetric: $S^{\mu\nu}_{\mathrm{L}} = -S^{\nu\mu}_{\mathrm{L}}$)
satisfy the Lorentz algebra (eq.~34.4):
\begin{equation}
  [S^{\mu\nu}_{\mathrm{L}},\, S^{\rho\sigma}_{\mathrm{L}}]
  = i\Bigl(
      g^{\nu\rho} S^{\mu\sigma}_{\mathrm{L}}
    - g^{\mu\rho} S^{\nu\sigma}_{\mathrm{L}}
    - g^{\nu\sigma} S^{\mu\rho}_{\mathrm{L}}
    + g^{\mu\sigma} S^{\nu\rho}_{\mathrm{L}}
    \Bigr),
  \tag{34.4}
\end{equation}
with metric $g^{\mu\nu} = \mathrm{diag}(+1,-1,-1,-1)$.

\subsection{Explicit matrices}

\paragraph{Pauli matrices.}
\begin{equation}
  \sigma_1 = \begin{pmatrix}0 & 1 \\ 1 & 0\end{pmatrix},\qquad
  \sigma_2 = \begin{pmatrix}0 & -i \\ i & 0\end{pmatrix},\qquad
  \sigma_3 = \begin{pmatrix}1 & 0 \\ 0 & -1\end{pmatrix}
  \tag{34.8}
\end{equation}

\paragraph{Spatial rotation generators (eq.~34.9).}
\begin{equation}
  {(S^{ij}_{\mathrm{L}})_a}^b = \tfrac{1}{2}\varepsilon^{ijk}\sigma_k
  \tag{34.9}
\end{equation}
\begin{align*}
  S^{12}_{\mathrm{L}} &= \begin{pmatrix}\tfrac{1}{2} & 0 \\ 0 & -\tfrac{1}{2}\end{pmatrix},&
  S^{13}_{\mathrm{L}} &= \begin{pmatrix}0 & \tfrac{i}{2} \\ -\tfrac{i}{2} & 0\end{pmatrix},&
  S^{23}_{\mathrm{L}} &= \begin{pmatrix}0 & \tfrac{1}{2} \\ \tfrac{1}{2} & 0\end{pmatrix}
\end{align*}

\paragraph{Boost generators (eq.~34.10).}
\begin{equation}
  {(S^{k0}_{\mathrm{L}})_a}^b = \tfrac{i}{2}\sigma_k
  \tag{34.10}
\end{equation}
\begin{align*}
  S^{10}_{\mathrm{L}} &= \begin{pmatrix}0 & -\tfrac{i}{2} \\ -\tfrac{i}{2} & 0\end{pmatrix},&
  S^{20}_{\mathrm{L}} &= \begin{pmatrix}0 & -\tfrac{1}{2} \\ \tfrac{1}{2} & 0\end{pmatrix},&
  S^{30}_{\mathrm{L}} &= \begin{pmatrix}-\tfrac{i}{2} & 0 \\ 0 & \tfrac{i}{2}\end{pmatrix}
\end{align*}

\noindent\textbf{Physical interpretation.}
The $i$ factor in the boost generators means boosts are \emph{not unitary}:
the Lorentz group is non-compact.  Rotations ($S^{ij}$) are Hermitian;
boosts ($S^{k0}$) are anti-Hermitian.

\paragraph{Numerical verification.}
\verified{All $\binom{6}{2} = 15$ commutator pairs satisfy eq.~(34.4). Max error: $0.0e+00$.}

%% ============================================================
\section{Right-Handed Spinor Field
         \texorpdfstring{$\psidag_{\dot{a}}$}{psi-dagger}}
%% ============================================================

\subsection{Why hermitian conjugation flips the representation}

For $\psi_a$ in $(2,1)$:
$N_i$ acts as $\tfrac{1}{2}\sigma_i$ (spin-$\tfrac{1}{2}$),
$N^\dagger_i$ acts trivially (spin-$0$).
Taking $\dagger$ swaps $N_i \leftrightarrow N^\dagger_i$.
Therefore $(\psi_a)^\dagger \equiv \psidag_{\dot{a}}$ lives in $(1,2)$.
\begin{equation}
  [\psi_a(x)]^\dagger = \psidag_{\dot{a}}(x).
  \tag{34.11}
\end{equation}

\noindent In Cadabra2: $\psi^{\dagger}\,_{\dal}$

\subsection{Right-handed generators and the dotted-index rule}

The dotted index $\dot{a}$ signals membership in $(1,2)$.
The generators satisfy (eq.~34.17):
\begin{equation}
  \boxed{(S^{\mu\nu}_{\mathrm{R}})_{\dot{a}}{}^{\dot{b}}
  = -\bigl[(S^{\mu\nu}_{\mathrm{L}})_a{}^b\bigr]^*}
  \tag{34.17}
\end{equation}

This has a physical consequence:
\begin{itemize}
  \item \textbf{Rotation generators} ($S^{ij}$): real parts unchanged,
        imaginary parts flip sign.  Since $\sigma_1, \sigma_3$ are real
        and $\sigma_2$ is purely imaginary,
        $S^{ij}_{\mathrm{R}} = -[S^{ij}_{\mathrm{L}}]^*$ differs from
        $S^{ij}_{\mathrm{L}}$ by the sign of $\sigma_2$ components.
  \item \textbf{Boost generators} ($S^{k0}$): the $i$ flips sign,
        so $S^{k0}_{\mathrm{R}} = -[S^{k0}_{\mathrm{L}}]^* =
        -\tfrac{i}{2}\sigma_k$.
        Boosts are reversed --- consistent with parity $L \leftrightarrow R$.
\end{itemize}

\paragraph{Explicit $S^{\mu\nu}_{\mathrm{R}}$ matrices.}
\begin{align*}
  S^{12}_{\mathrm{R}} &= \begin{pmatrix}-\tfrac{1}{2} & 0 \\ 0 & \tfrac{1}{2}\end{pmatrix},&
  S^{13}_{\mathrm{R}} &= \begin{pmatrix}0 & \tfrac{i}{2} \\ -\tfrac{i}{2} & 0\end{pmatrix},&
  S^{23}_{\mathrm{R}} &= \begin{pmatrix}0 & -\tfrac{1}{2} \\ -\tfrac{1}{2} & 0\end{pmatrix}\\[4pt]
  S^{10}_{\mathrm{R}} &= \begin{pmatrix}0 & -\tfrac{i}{2} \\ -\tfrac{i}{2} & 0\end{pmatrix},&
  S^{20}_{\mathrm{R}} &= \begin{pmatrix}0 & \tfrac{1}{2} \\ -\tfrac{1}{2} & 0\end{pmatrix},&
  S^{30}_{\mathrm{R}} &= \begin{pmatrix}-\tfrac{i}{2} & 0 \\ 0 & \tfrac{i}{2}\end{pmatrix}
\end{align*}

%% ============================================================
\section{The \texorpdfstring{$\varepsilon$}{epsilon} Symbol ---
         SL(2,\texorpdfstring{$\mathbb{C}$}{C}) Metric}
%% ============================================================

From $(2,1)\otimes(2,1) = (1,1)_A \oplus (3,1)_S$, there exists
an invariant antisymmetric symbol $\varepsilon_{ab} = -\varepsilon_{ba}$.
In Cadabra2: $\epsilon_{\alpha \beta}$.

\subsection{Normalization (Srednicki convention, eq.~34.22)}
\begin{equation}
  \varepsilon^{12} = \varepsilon_{21} = +1,\qquad
  \varepsilon^{21} = \varepsilon_{12} = -1.
\end{equation}
\[
  \varepsilon_{ab} = \begin{pmatrix}0 & -1 \\ 1 & 0\end{pmatrix},\qquad
  \varepsilon^{ab} = \begin{pmatrix}0 & 1 \\ -1 & 0\end{pmatrix}
\]
Completeness (eq.~34.23):
\begin{equation}
  \varepsilon_{ab}\,\varepsilon^{bc} = \delta_a{}^c.
\end{equation}

\subsection{Raising and lowering}
\begin{align}
  \psi^a &= \varepsilon^{ab}\,\psi_b
  &&\text{(Cadabra2: } \epsilon^{\alpha \beta} \psi_{\beta}\text{)} \\
  \psi_a &= \varepsilon_{ab}\,\psi^b
\end{align}

\noindent\textbf{Sign trap (eq.~34.27):}
\begin{equation}
  \psi^a\chi_a = \varepsilon^{ab}\psi_b\chi_a
  = -\varepsilon^{ba}\psi_b\chi_a = -\psi_b\chi^b.
\end{equation}
The contraction $\psi^a\chi_a = -\psi_a\chi^a$ carries an essential minus sign.
The same $\varepsilon_{\dot{a}\dot{b}}$ structure holds for dotted indices.

\paragraph{Numerical verification.}
\verified{$\varepsilon_{ab}\,\varepsilon^{bc} = \delta_a{}^c$ and invariance under SL(2,$\mathbb{C}$): max error $0.0e+00$.}

%% ============================================================
\section{Lorentz-Invariant Spinor Products}
%% ============================================================

\subsection{Left-handed (``angle bracket'')}
\begin{equation}
  \langle\psi\chi\rangle
  \equiv \varepsilon^{\alpha\beta}\psi_\alpha\chi_\beta
  = \psi^\alpha\chi_\alpha
  \tag{35.21 preview}
\end{equation}
Cadabra2: $\epsilon^{\alpha \beta} \psi_{\alpha} \chi_{\beta}$.

Antisymmetry (Grassmann + $\varepsilon$ antisymmetric):
$\langle\psi\chi\rangle = -\langle\chi\psi\rangle$.

\subsection{Right-handed (``square bracket'')}
\begin{equation}
  [\psidag\chidag]
  \equiv \varepsilon_{\dot{\alpha}\dot{\beta}}
         \psidag^{\dot{\alpha}}\chidag^{\dot{\beta}}
\end{equation}
Cadabra2: $\epsilon_{\dal \dbe} \bar{\psi}\,^{\dal} \bar{\chi}\,^{\dbe}$.

%% ============================================================
\section{The \texorpdfstring{$\sigma^\mu$}{sigma\^mu} Symbol ---
         Vector/Spinor Dictionary}
%% ============================================================

A field $A_{a\dot{a}}$ in $(2,2)$ maps to a 4-vector via (eq.~34.28):
\begin{equation}
  A_{a\dot{a}} = \sigma^\mu_{a\dot{a}}\,A_\mu,\qquad
  \sigma^\mu_{a\dot{a}} = (I,\,\vec{\sigma}).
  \tag{34.30}
\end{equation}
\[
  \sigma^0 = \begin{pmatrix}1 & 0 \\ 0 & 1\end{pmatrix},\quad
  \sigma^1 = \begin{pmatrix}0 & 1 \\ 1 & 0\end{pmatrix},\quad
  \sigma^2 = \begin{pmatrix}0 & -i \\ i & 0\end{pmatrix},\quad
  \sigma^3 = \begin{pmatrix}1 & 0 \\ 0 & -1\end{pmatrix}
\]
and $\bar{\sigma}^\mu_{\dot{a}a} = (I,-\vec{\sigma})$:
\[
  \bar{\sigma}^0 = \begin{pmatrix}1 & 0 \\ 0 & 1\end{pmatrix},\quad
  \bar{\sigma}^1 = \begin{pmatrix}0 & -1 \\ -1 & 0\end{pmatrix},\quad
  \bar{\sigma}^2 = \begin{pmatrix}0 & i \\ -i & 0\end{pmatrix},\quad
  \bar{\sigma}^3 = \begin{pmatrix}-1 & 0 \\ 0 & 1\end{pmatrix}
\]

\paragraph{Key identity.}
\begin{equation}
  \operatorname{tr}(\sigma^\mu\bar{\sigma}^\nu)
  \equiv \sigma^\mu_{a\dot{a}}\,\bar{\sigma}^{\nu\,\dot{a}a}
  = 2\,g^{\mu\nu}.
\end{equation}
\verified{Verified numerically: max error $0.0e+00$.}

%% ============================================================
\section{Summary}
%% ============================================================

\begin{center}
\renewcommand{\arraystretch}{1.4}
\begin{tabular}{@{}ll@{}}
  \toprule
  Object & Expression \\
  \midrule
  Left-handed field & $\psi_\alpha$ \quad (undotted index) \\
  Right-handed field & $\psidag_{\dot{\alpha}} = (\psi_\alpha)^\dagger$
                       \quad (dotted index) \\
  Rotation generator & $(S^{ij}_L)_a{}^b = \tfrac{1}{2}\varepsilon^{ijk}\sigma_k$ \\
  Boost generator    & $(S^{k0}_L)_a{}^b = \tfrac{i}{2}\sigma_k$ \\
  R-generators       & $S^{\mu\nu}_R = -[S^{\mu\nu}_L]^*$ \\
  Raise index        & $\psi^a = \varepsilon^{ab}\psi_b$ \\
  Lower index        & $\psi_a = \varepsilon_{ab}\psi^b$ \\
  Sign identity      & $\psi^a\chi_a = -\psi_a\chi^a$ \\
  Angle bracket      & $\langle\psi\chi\rangle = \varepsilon^{\alpha\beta}\psi_\alpha\chi_\beta$ \\
  Square bracket     & $[\psidag\chidag] = \varepsilon_{\dot\alpha\dot\beta}
                        \psidag^{\dot\alpha}\chidag^{\dot\beta}$ \\
  Vector dictionary  & $\sigma^\mu_{a\dot{a}} = (I,\vec\sigma)$,\quad
                        $\bar\sigma^\mu_{\dot{a}a} = (I,-\vec\sigma)$ \\
  Trace identity     & $\operatorname{tr}(\sigma^\mu\bar\sigma^\nu) = 2g^{\mu\nu}$ \\
  \bottomrule
\end{tabular}
\end{center}

\bigskip
\noindent\textbf{Next:} Chapter 35 develops the index-free dot/bar notation,
derives the $\sigma$-algebra identities, and constructs the Weyl Lagrangian.

\end{document}